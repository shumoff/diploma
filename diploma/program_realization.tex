\subsection{Выбор инструментов разработки}\label{subsec:development_tools}
Рекомендательные алгоритмы реализовывались на языке Python\cite{python}.
Выбор обусловлен наличием большого количества библиотек для работы с математическими вычислениями, с алгоритмами машинного обучения и нейросетями, а также лёгкость прототипирования на нём.
Основные используемые библиотеки:
\begin{itemize}
\item \textbf{NumPy}\cite{numpy} -- для работы с матричными вычислениями
\item \textbf{pandas}\cite{pandas} -- для работы с большими объёмами данных, в том числе для препроцессинга
\item \textbf{scikit-learn}\cite{scikit-learn} -- для организации обучения моделей машинного обучения: разбиение данных на обучающую и тестовую выборки, вычисление метрик
\item \textbf{Matplotlib}\cite{matplotlib} -- для построения графиков
\item \textbf{TensorFlow}\cite{tensorflow} -- для обучения нейросетевых моделей
\end{itemize}

Python -- интерпретируемый язык.
Эта категория языков, как правило, медленнее компилируемых.
Однако, библиотеки, реализующие сложные математические вычисления, написаны на компилируемых языках, таких как С\cite{c-language} или C++\cite{c++}.
Это делает код, использующий эти библиотеки, очень быстрым.
Таким образом, в данном контексте выбор Python в качестве языка программирования упрощает разработку, сохраняя при этом производительность программ.

\pagebreak
\subsection{Структура проекта}\label{subsec:project_structure}
Каждый рекомендательный алгоритм представляет из себя отдельную модель (класс).
Все модели наследуются от базового класса, в котором имплементировано большое количество рутины, возникающей в ходе обучения:
\begin{itemize}
\item Препроцессинг данных
\item Инициализация модели
\item Разбиение данных на обучающую и тестовую выборки
\item Построение графиков и мониторинг процесса обучения
\item Сохранение параметров обучения модели и её выхода
\item Оценка качества модели
\item Поиск гиперпараметров (таких как размерность векторных представлений или количество эпох обучения), дающих наилучшее качество
\end{itemize}

Каждый класс содержит в себе программную реализацию конкретного алгоритма.
В отдельных пакетах реализуются загрузка данных и метрики оценивания.

Процесс обучения модели в общем виде представлен алгоритмом \ref{alg:learning-process}:

\vspace{1em}
\begin{algorithm}[H]
 \SetAlgoLined
 \KwResult{обученная модель}
 инициализировать модель\;
 \While{эпоха обучения != максимум эпох}{
  сделать шаг алгоритма\;
  вычислить метрики\;
 }
 вычислить итоговое качество\;
 \caption{Процесс обучения}\label{alg:learning-process}
\end{algorithm}

\vspace{1em}
Самым важным на этапе исследования является подбор гиперпараметров модели, приводящих к оптимальному качеству.
Поиск оптимальных гиперпараметров представлен алгоритмом \ref{alg:params-tuning}:

\vspace{1em}
\begin{algorithm}[H]
 \SetAlgoLined
 \KwResult{наилучший набор гиперпараметров}
 \ForEach{набор гиперпараметров}{
     инициализировать модель\;
     обучить модель\;
     вычислить итоговое качество\;
 }
 сравнить качество разных наборов гиперпараметров\;
 \caption{Поиск оптимальных гиперпараметров}\label{alg:params-tuning}
\end{algorithm}
